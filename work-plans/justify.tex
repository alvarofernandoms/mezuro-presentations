\documentclass[12pt]{article}

\usepackage{url}
\usepackage[brazil]{babel}
\usepackage[utf8]{inputenc}
\usepackage{float}
\usepackage{setspace}
\usepackage{tabu}

\begin{document}
  \title{Projeto Mezuro\\
         Justificativa para Bolsistas}

  \maketitle

  O projeto Mezuro procura oferecer uma plataforma de software livre para automatizar e unificar a coleta, visualização e manutenção do histórico de métricas estáticas de código fonte. E, com isso, difundir seu uso como uma ferramenta auxiliar no processo de desenvolvimento de software. Então a plataforma possibilitará estudos estatísticos sobre a relevância de cada métrica e evolução de grandes projetos de software livre ao longo do tempo.

  Com esse objetivo o projeto Mezuro teve seu potencial reconhecido no Instituto de Matemática e Estatística da Universidade de São Paulo como parte de disciplinas ministradas, dissertação de mestrado do ex-aluno Carlos Morais e tese de doutorado do ex-aluno Paulo Meirelles. Além das diversas bolsas já concedidas pelo projeto NAPSoL anteriormente e em vigor. E, mais recentemente, como parte integrante do novo Portal do Software Público Brasileiro (projeto do Ministério do Planejamento do Governo Federal), sob o agregador de sistemas Colab.

  Assim, o projeto Mezuro se mostra como uma ferramenta inédita para a comunidade de software livre cujo valor vem sido reconhecido em diferentes escopos e cujo potencial para pesquisa ainda pode ser explorado quando possuir dados suficientes processados.

\end{document}
