\section{Conclusions}
\label{sec:conclusions}

This paper presented the evolution of Mezuro -- a free software platform that
is part of a new generation of source code metrics analysis tools.  It has
useful features for both software engineer and researcher working with source
code analysis. Mezuro is fully flexible allowing easy integration with distinct
source code collectors. Also, it provides an environment where software
engineers can define their own threshold configurations, according to a software
implementation context and their experiences. These thresholds are shared among
other software engineers on the Mezuro plaform.  In short, each metric loaded
can support as many thresholds as possible to provide a full interpretation
about what a metric value means.

To support its requirements and work on real-world projects, Mezuro architecture
migrated from a service-based implemention to a micro-service cloud-based
implementation. The final step on the evolution of the Mezuro project brings
it to the new era of cloud computing with scalability, distributed processing,
and fault tolerance through the described division into smaller services. That
modularization enables the system to scale each of its components on demand.
As result of the described evolution through time here are summarized the main
features of Mezuro:

\begin{itemize}

  \item Latest standards on web development (through Boilerplate and
        TwitterBootstrap frameworks) brings a clean interface, easy to use, and familiar
        to others found by users on the web;

  \item Multi-user environment with permission management which allows the
        collaboration and spreading of standards;

  \item Historical data about the evolution of metric values through time;

  \item Multi-language support for analysis;

  \item Configurable set of metrics and interpretations which enables it to
        suit better to the variety of projects in the world;

  \item Distributed services with specific responsibilities which makes them
        easier to extend and install on cloud infrastructures.

\end{itemize}

As well it provides to developers the flexibility necessary for expansions such as
adding support for collecting source code metrics from other programming
languages (such as Ruby, Python, and PHP), which is the current effort of our team.

%%%%%%%%%%%%%%%
\begin{comment}

\section*{Acknowledgment}

The authors of this paper were supported by Brazilian Nation Research Council
(CNPq) and Ministry of Planning, Budget, and Management (MP) of Brazil. 
%
The Mezuro has been developed as a project of the FLOSS Competence Center
(CCSL) at University of São Paulo (USP) and it is part of the new Brazilian
Public Software (SPB) Portal.
%
The authors would like to thank Dr. Alfredo Goldman for his collaborations
through the eXtreme Programming Laboratory course.
%
Also, a special thanks to other (previous and current) Mezuro oficial developers:
Alessandro Palmeira,
Carlos Morais,
Daniel Paulino,
Fellipe Souto,
Guilherme Rojas,
Heitor Reis,
João da Silva, and
Renan Fichberg.

\end{comment}
%%%%%%%%%%%%%
