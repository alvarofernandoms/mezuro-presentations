\section{Conclusions}
\label{sec:conclusions}

%TODO: Conclusions (½ página)
%% pequeno resumo geral da contribuição
%% trabalhos futuros e problemas em aberto para o futuro

This paper presented the evolution of Mezuro -- a free software platform that
is part of a new generation of source code metrics analysis tools.  It has
useful features for both software engineer and researcher working with source
code analysis. Mezuro is fully flexible allowing easy integration with distinct
source code collector. Also, it provides an environment where software
engineers can define their own threshold configurations, according to software
implementation context and their experiences. These thresholds are shared among
other software engineers on the Mezuro plaform.  In short, each metric loaded
can support as many thresholds as possible to provide a full interpretation
about what a metric value means.

Future works include the support of other metric collector tools, especially
to provide Python and Ruby source code analysis.

TODO: ... evolution and cloud computing context

\section*{Acknowledgment}

The authors of this paper were supported by Brazilian Nation Research Council
(CNPq) and Ministry of Planning, Budget, and Management (MP) of Brazil. 
%
The Mezuro has been developed as a project of the FLOSS Competence Center
(CCSL) at University of São Paulo (USP) and it is part of the new Brazilian
Public Software (SPB) Portal.
%
The authors would like to thank Dr. Alfredo Goldman for his collaborations
through the eXtreme Programming Laboratory course.
%
Also, a special thanks to other (previous and current) Mezuro oficial developers:
Alessandro Palmeira,
Daniel Paulino,
Fellipe Souto,
Guilherme Rojas,
Heitor Reis,
João da Silva, and
Renan Fichberg.
