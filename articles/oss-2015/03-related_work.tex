\section{Related Works}
\label{sec:related_works}

% Related Work (1,5 página)
%% ferramentas mais básicas e pioneiras de métricas de código-fonte
%% ferramentas (Sonar, Jenkins e CodeClimate)
%% a descrição tem que abordar a questão da nuvem

\subsection{Source code analysis tools}
\label{subsec:related-tools}

We have studied about 11 FLOSS source code analysis tools:
%
Analizo,
CCCC,
Checkstyle,
CMetrics,
CPPX,
Cscope,
CTAGX,
LDX,
JaBUTi,
MacXim,
and Metrics (Eclipse plug-in).
%
Also, we have defined the following requirements to work on source
code metrics:

\begin{itemize}

  \item The tool should support source code metrics \textbf{thresholds} to
provide different interpretations about metric values.

  \item The tool should support the analysis of different programming languages
(\textbf{multi-language}).

  \item The tool should provide clear interfaces for adding new metrics and supporting
different programming language (\textbf{extensibility}).

  \item The tool should be \textbf{free software}, available without
restrictions to allow other researchers to replicate our studies and results
fully.

  \item The tool should be supported by active developers who know the tool
architecture (\textbf{maintained}).
 
\end{itemize}


\begin{table}[htb]
  \centering
%  \scalefont{0.95}
  \begin{tabular}{|l|c|c|c|c|}
    \hline
    \textbf{Tools} &
    \textbf{Languages} & %Multi-language
    \textbf{Extensible} & %Extensibility
    \textbf{Thresholds} & % To Support Thresholds
    \textbf{Maintained} \\\hline\hline %Actively maintained 
				    
    Analizo 	& C, C++, Java & Yes & No  & Yes    \\\hline
    CCCC 	& C++, Java    & No  & No  & Yes    \\\hline
    Checkstyle	& Java         & Yes  & Yes & Yes    \\\hline
    CMetrics	& C            & Yes & No  & Yes    \\\hline
    CPPX	& C, C++       & No  & No  & No     \\\hline
    Cscope	& C            & No  & No  & Yes    \\\hline
    CTAGX	& C            & No  & No  & No     \\\hline
    LDX		& C, C++       & No  & No  & No     \\\hline
    JaBUTi	& Java         & No  & No  & Yes    \\\hline
    MacXim	& Java         & No  & No  & Yes    \\\hline
    Metrics 	& Java         & No  & Yes & Yes    \\\hline
    
  \end{tabular}


  \caption{Existing tools versus our defined requirements}
  \label{tab:tools}
%  \scalefont{1}
\end{table}
\vspace{-1em}

Finally, according to the Table \ref{tab:tools}, we compare all of these tools
to our requirements. So, the Analizo and Checkstyle are indicated as able to
analyze source code from our initial. Both have documented extension
interfaces: Analizo focused on C e C++ project and Checkstyle as a better
option for Java projects. In short,  at the first moment, Analizo and
Checkstyle act as Mezuro project default source code analysis tool since it
supports different source code collector tools.


\subsection{Related projects}
\label{subsec:related-projects}

In order to evaluate what is available, were gathered informations about similar tools already consolidated among software developers, with regard to the following criteria: interface that joins several tools available for metric collection; allow selection and composition of metrics in a flexible way; keeping record of the evolution history; friendly results view.

The first tool, and the one that is the closest to Mezuro, is SonarQube\footnote{\url{http://www.sonarqube.org/}}. Licensed under the LGLv3, it's a free software which offers a platform for code quality management through its plugins available from a library\footnote{\url{http://docs.sonarqube.org/display/SONAR/Plugin+Library}}. In its basic version, it classifies problems found in the code written on several languages and calculates simple test coverage metrics and technical debt. However, its best plugins are closed source programs which demands payment as well, like the C/C++ plugin\footnote{\url{http://www.sonarsource.com/products/plugins/languages/cpp/}}.

Following, Code Climate \footnote{\url{https://codeclimate.com/}} is a online platform that provides source code quality analysis and code coverage monitoring for free for open source Ruby, JavaScript and PHP\footnote{at the time of publishing this article PHP support was under beta} available through public Git repositories. Basically the software searches through the user program for ``code smells'' and classifies the ones found accordingly to their method size and block duplication. As it finds the code issues, it assigns values to the code so in the end it can account grades, from A to F, using these values. Notice that this kind of analysis will catch portions of the code that were conscientious architecture decisions made by the developers and here is the importance of flexible metric selection from one project to another. Recently it published a preliminary statistical analysis of all the projects\footnote{\url{http://blog.codeclimate.com/blog/2014/05/21/does-team-size-impact-code-quality/?utm_source=Code+Climate&utm_campaign=69c024549d-newsletter-NI-2014-05-22&utm_medium=email&utm_term=0_672a7f5529-69c024549d-317410425}} but it is still far from a good historical visualization.

A final remark on this analysis of similar projects is to mention projects analyzed by Jenkins. It is a free software mainly used for continuous integration which has a rich environment of plugins where some of them provide metrics like test coverage and basic complexity metrics. But none of them providing interpretation of the results, any kind of flexibility on metric definition and laking the historical visualization of code metrics results.
