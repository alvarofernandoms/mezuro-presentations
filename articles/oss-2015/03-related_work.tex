\section{Related Works}
\label{sec:related_works}

Initially, as the first step for designing Mezuro, we studied several source
code analysis tools, such as
%
Analizo~\cite{analizo}, CCCC~\footnote{\url{cccc.heckstyle.sourceforge.net}},
Checkstyle~\footnote{\url{checkstyle.sourceforge.net}},
CMetrics~\footnote{\url{tools.libresoft.es/cmetrics}}, CPPX~\cite{hassan2005},
Cscope~\footnote{\url{cscope.sourceforge.net}}, CTAGX~\cite{hassan2005},
LDX~\cite{hassan2005} JaBUTi~\cite{jabuti}, and Metrics (Eclipse
plug-in)~\footnote{\url{metrics.sourceforge.net}}.
%
So, we have defined the following requirements to select them: should
support source code metrics thresholds to provide different interpretations
about metric values (\textbf{thresholds}); should support the analysis of different programming
languages (\textbf{multi-language}); should provide clear interfaces for adding new metrics and
supporting different programming language (\textbf{extensible}); should be free software,
available without restrictions to allow other researchers to replicate our
studies and results fully (\textbf{FLOSS}); and should be supported by active developers who
know the tool architecture (\textbf{maintained}).

\begin{table}[htb]
  \centering
\scalefont{0.9}
  \begin{tabular}{|l|c|c|c|c|}
    \hline
    \textbf{Tools} &
    \textbf{Languages} & %Multi-language
    \textbf{Extensible} & %Extensibility
    \textbf{Thresholds} & % To Support Thresholds
    \textbf{Maintained} \\\hline\hline %Actively maintained

    Analizo 	& C, C++, Java & Yes & No  & Yes    \\\hline
    CCCC 	& C++, Java    & No  & No  & Yes    \\\hline
    Checkstyle	& Java         & Yes  & Yes & Yes    \\\hline
    CMetrics	& C            & Yes & No  & Yes    \\\hline
    CPPX	& C, C++       & No  & No  & No     \\\hline
    Cscope	& C            & No  & No  & Yes    \\\hline
    CTAGX	& C            & No  & No  & No     \\\hline
    LDX		& C, C++       & No  & No  & No     \\\hline
    JaBUTi	& Java         & No  & No  & Yes    \\\hline
    MacXim	& Java         & No  & No  & Yes    \\\hline
    Metrics 	& Java         & No  & Yes & Yes    \\\hline

  \end{tabular}
  \caption{Existing tools versus our defined requirements}
  \label{tab:tools}
\scalefont{1}
\end{table}

As shown by Table~\ref{tab:tools}, we compare all of those FLOSS tools according to the above requirements. In short, the Analizo and Checkstyle are indicated as able to
analyze source code meeting our initial requirements. Both have documented
extension interfaces: Analizo focused on C e C++ project and Checkstyle as a
better option for Java projects. At the first moment, Analizo and Checkstyle
act as Mezuro project default source code analysis tool.

Secondly, during the development of the Mezuro project,
we have gathered information about platforms similar to Mezuro already
consolidated among software developers with regard to the following
criteria: interface that joins several tools available for calculating metrics
(\textbf{collectors}); allow selection and composition of metrics in a flexible
way (\textbf{composition}); keeping record of the evolution history
(\textbf{history}); friendly results view (\textbf{graphics}); available as
service on the web (\textbf{on-line}).


\begin{table}[htb]
  \centering
\scalefont{0.9}
  \begin{tabular}{|l|c|c|c|c|}
    \hline
    Criteria                   & SonarQube & CodeClimate & HackyStat & Mezuro \\\hline\hline

    \textbf{Multi-Language}    & Yes       & Yes         & No        & Yes   \\\hline
    \textbf{Collectors}        & Yes       & --          & No        & Yes   \\\hline
    \textbf{Thresholds}        & Yes       & Yes         & Yes       & Yes   \\\hline
    \textbf{Composition}       & Yes       & No          & No        & Yes   \\\hline
    \textbf{Graphics}          & Yes       & Yes         & Yes       & Yes   \\\hline
    \textbf{History}           & Yes       & Yes         & Yes       & Yes   \\\hline
    \textbf{On-line}           & No        & Yes         & No        & Yes   \\\hline
    \textbf{FLOSS}             & Yes/No    & No          & Yes       & Yes   \\\hline
    \textbf{Maintained}        & Yes       & Yes         & No        & Yes   \\\hline

  \end{tabular}
  \caption{Our criteria versus Existing platforms}
  \label{tab:platforms}
\scalefont{1}
\end{table}

The closest platform to Mezuro in terms of characteristics showed in Table~\ref{tab:platforms},
SonarQube~\footnote{\url{sonarqube.org}} is a free software which offers a
platform for code quality management through its plugins available from a
library. In its basic version, it classifies problems found in the code written
on several languages and calculates simple test coverage metrics and technical
debt. However, its best plugins are proprietary programs which demands payment
as well, such as the C/C++ plugin.

Code Climate~\footnote{\url{codeclimate.com}} is an online
platform that provides source code quality analysis and code coverage
monitoring for FLOSS Ruby, JavaScript and PHP~\footnote{At the
time of publishing this article PHP support was under beta.} available through
public Git repositories. Basically the software searches through the user
program for ``code smells'' and classifies the ones found accordingly to their
method size and block duplication. As it finds the code issues, it assigns
values to the code so in the end it can account grades, from A to F, using
these values. Notice that this kind of analysis will catch portions of the code
that were conscientious architecture decisions made by the developers and here
is the importance of flexible metric selection from one project to another.
Recently it published a preliminary statistical analysis of all the
projects~\footnote{\url{bit.ly/1zpcdlR}}
but it is still far from a good historical visualization.

Hackystat~\footnote{\url{hackystat.org}} is a FLOSS framework for
automated collection, analysis, and visualization of process and product data.
It was developed from 2001 until 2010, according to its repository activities.
Similar to Mezuro, its architecture was evolved from a client-server web
application to SOA using REST and HTTP protocol for communication among
services~\cite{chauhan2011}.  To use Hackystat, developers need to install an
Eclipse IDE plugin to collect process and product data. Another client is
called SensorShell, which can also send information to Hackystat server.

In general, Mezuro have more features than other similar platforms, as showed
in Table~\ref{tab:platforms}. For example, Code Climate is an excellent
platform with a good approach but it is not a free software. HackyStat is
deprecated and has several limitations, such as only analysing Java code. Code
Climate collects metrics from Ruby, JavaScript, and PHP, while the vast
majority of free software applications is written in C~\cite{robles2006},
followed by C++ and Java as we observed on the OpenHub
website~\footnote{\url{openhub.net}}. Mezuro collects and analyses metrics from
C, C++, and Java source code. Finally, despite SonarQube has a lot of features,
it is not a platform available on the Web. Its users must install it in their
own server. Anyway, the worst problem from SonarQube is its proprietary plugin
to a better use of it.

A final remark on this analysis of similar projects is to mention projects
analysed by Jenkins~\footnote{\url{jenkins-ci.org}} and
Travis~\footnote{\url{jenkins-ci.org}}. They are free software solutions
mainly used for continuous integration which have a rich environment of plugins
where some of them provide metrics like test coverage and basic complexity
metrics. But none of them provide interpretation of the results, any kind of
flexibility on metric definition and lack the historical visualization of
code metrics results. In other words, continuous integration tools work on
another scope. They are not comparable to Mezuro, SonarQube and CodeClimate.
