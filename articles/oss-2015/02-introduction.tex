\section{Introduction}
\label{introduction}

Whatever the methodology, from a practical point of view, software development
should be guided by an aspects to control the software quality in the long
term: source code quality~\cite{martin2008}. Metrics can help software
engineers to observe the source code quality~\cite{SEI88}. Software Engineering
requires the understanding of software, which is the result from the writing of
source code~\cite{martin2008}. Also, we argue that software engineers and
researchers need to analyze source codes to understand software projects.

For free software~\footnote{We consider the terms ``free software'', ``open
source software'' (OSS), and  ``free/libre/open source software'' (FLOSS)
equivalent.} communities, source code is the main artefact of software
development activities since features should be constantly released to users.
In fact, free software source code is written gradually and different
developers make updates as well as improvements on an ongoing
basis~\cite{martin2008}. Thus, new features are inserted and bugs are resolved
during software development and maintenance iterations.

Software engineers make decisions when are programming at the method and class
level, influencing the source code quality~\cite{beck2007}. To make the best
decisions, we argue they should track attributes of their source codes from an
automated and objective way to interpret metric values. Even with the fact that
source code metrics have been proposed since the 1970s~\cite{SEI88}, there is
not a set of standard measures established for them.  Moreover, we have
observed there is not a systematic approach to use, interpret, and understand
source code metrics. We have also identified limitations from several FLOSS
tools to monitor this kind of metrics, such as for collecting automatically
source code metrics values independent of the programming language, for
interpreting measurement results associating them with source code quality, and
an avaliable and dependable service on the Cloud computing.

This paper presents the evolution of Mezuro~\cite{mezuro2012}, a free software
platform designed to incorporate any source code metric tool as well as provide
an approach to explore source code metrics on the Web. Mezuro has been
developed since 2009. From 2014, we are refactoring it to work on according to
the Cloud computing concepts and infrastructure~\cite{louridas2010}.

Mezuro was designed to collect, analyze, and store metric values from source
code of large free software projects, sharing the results on the a
social-technical networking environment~\cite{mezuro2012}. In particular, to
support its scalability requirements, Mezuro architecture needed to migrate
from a web service implementation based on SOAP communication to a
micro-service architecture~\cite{namiot2014micro} based on RESTful
communication. In this paper, we describe how to we have built this new
distributed architecture to bring Mezuro to the new era of Cloud computing with
scalability, distributed processing, and fault tolerance through smaller
services. Ultimately, we are releasing on the Web a free software platform to
track source code metrics from real-world projects.

The remainder of this paper is organized as follows. Section
\ref{sec:related_works} describes the related projects and works. Section
\ref{sec:mezuro} presents the Mezuro project and how its implementation evolved
from a desktop application to a distributed system that runs on the Cloud. At
last, on Section \ref{sec:conclusions}, we conclude the paper and discuss
future work.
